% !TeX root = ../main.tex
% -*- coding: utf-8 -*-

\chapter{理论方法}
\label{ch3}

\section{主方程}

\subsection{Liouville--von Neumann方程的解}
为了在模型中加入环境的影响,我们可以假设系统$S$和外部环境$R$的耦合具有如下形式的哈密顿量:
\begin{equation}
\mathcal{H}=H_{S}+H_{R}+H_{S R}
\label{Hsr}
\end{equation}
这里$H_{S}$和$H_{R}$分别是系统和环境的哈密顿量,$H_{SR}$为相互作用的哈密顿量。我们可以先不拘泥于环境哈密顿量的具体形式,只需要知道他可以表示为温度和能量的函数。我们的目标是在无需关注环境具体状态的情形下研究混合体系$S \otimes R$中目标系统$S$的演化。为了做到这一点,需要用到密度算符的工具。我们用$\chi(t)$表示$S \otimes R$的密度算符并对环境的部分求迹,定义约化密度算符为
\begin{equation}
\rho(t) \equiv \operatorname{tr}_{R}[\chi(t)]
\end{equation}
显然,如果我们有了约化密度算符$\rho(t)$,很容易就能算出系统$S$的Hilbert空间中任一算符$\hat{O}$的平均,而不需要知道$\chi(t)$的具体形式:
\begin{equation}
\langle\hat{O}\rangle=\operatorname{tr}_{S \otimes R}[\hat{O} \chi(t)]=\operatorname{tr}_{S}\left\{\hat{O} \operatorname{tr}_{R}[\chi(t)]\right\}=\operatorname{tr}_{S}[\hat{O} \rho(t)]
\end{equation}
接下来我们要做的是得到$\rho(t)$演化方程。

%Schr\"{o}dinger
已知$\chi$的Liouville--von Neumann方程为
\begin{equation}
\dot{\chi}=\frac{1}{i \hbar}[\mathcal{H}, \chi]
\label{LMEq1}
\end{equation}
这里可以把$\mathcal{H}$中的$H_{S}+H_{R}$看作主要部分,$H_{SR}$当成次要部分,转入相互作用绘景下来方便我们的处理:
\begin{equation}
\tilde{\chi}(t) \equiv e^{(i / \hbar)\left(H_{S}+H_{R}\right) t} \chi(t) e^{-(i / \hbar)\left(H_{S}+H_{R}\right) t}
\end{equation}
结合\eqref{Hsr}、\eqref{LMEq1}与上式后可以得到
\begin{align}
\dot{\tilde{\chi}} &=\frac{i}{\hbar}\left(H_{S}+H_{R}\right) \tilde{\chi}-\frac{i}{\hbar} \tilde{\chi}\left(H_{S}+H_{R}\right)+e^{(i / \hbar)\left(H_{S}+H_{R}\right) t} \dot{\chi} e^{-(i / \hbar)\left(H_{S}+H_{R}\right) t} \notag \\
&=\frac{1}{i \hbar}\left[\tilde{H}_{S R}(t), \tilde{\chi}\right]
\label{LMEq2}
\end{align}
% oOSRHH{\ttfamily oOSRHH}{\sffamily oOSRHH}
其中$\tilde{H}_{S R}(t)$是显含时间的量:
\begin{equation}
\tilde{H}_{S R}(t) \equiv e^{(i / \hbar)\left(H_{S}+H_{R}\right) t} H_{S R} e^{-(i / \hbar)\left(H_{S}+H_{R}\right) t}
\end{equation}
对方程\eqref{LMEq2}直接积分可以得到
\begin{equation}
\tilde{\chi}(t)=\chi(0)+\frac{1}{i \hbar} \int_{0}^{t} d t^{\prime}\left[\tilde{H}_{S R}\left(t^{\prime}\right), \tilde{\chi}\left(t^{\prime}\right)\right]
\end{equation}
把上式回代到方程\eqref{LMEq2}的等号右侧:
\begin{equation}
\dot{\tilde{\chi}}=\frac{1}{i \hbar}\left[\tilde{H}_{S R}(t), \chi(0)\right]-\frac{1}{\hbar^{2}} \int_{0}^{t} d t^{\prime}\left[\tilde{H}_{S R}(t),\left[\tilde{H}_{S R}\left(t^{\prime}\right), \tilde{\chi}\left(t^{\prime}\right)\right]\right]
\label{LMExpd2}
\end{equation}
以上的推导都是严格的,我们甚至可以重复\eqref{LMEq2}到\eqref{LMExpd2}的过程来得到无限多的积分项,但是这里做到二阶就足够引入我们下面的近似了。

\subsection{Born--Markov近似}
假设$t=0$时的系统$S$和环境$R$之间没有关联但是相互作用已然存在,我们可以把$\chi(0)=\tilde{\chi}(0)$表示为
\begin{equation}
\chi(0)=\rho(0) R_{0}
\end{equation}
其中$R_{0}$是初始时刻环境的密度矩阵。记$\tilde{\chi}(t)$求迹后的密度矩阵为$\tilde{\rho}(t)$:
\begin{equation}
\operatorname{tr}_{R}[\tilde{\chi}(t)]=e^{(i / \hbar) H_{S} t} \rho(t) e^{-(i / \hbar) H_{S} t} \equiv \tilde{\rho}(t)
\label{InitialAssumption}
\end{equation}
对方程\eqref{LMExpd2}进行求迹运算后我们就可以得到如下的主方程:
\begin{equation}
\dot{\tilde{\rho}}=-\frac{1}{\hbar^{2}} \int_{0}^{t} d t^{\prime} \operatorname{tr}_{R}\left\{\left[\tilde{H}_{S R}(t),\left[\tilde{H}_{S R}\left(t^{\prime}\right), \tilde{\chi}\left(t^{\prime}\right)\right]\right]\right\}
\label{MasterEq}
\end{equation}
这里我们已经通过假设$\operatorname{tr}_{R}\left[\tilde{H}_{S R}(t) R_{0}\right]=0$消去了\eqref{LMExpd2}中的第一项。这项假设需要相互作用$H_{S R}$在$R_{0}$态下的平均值为零,我们总是可以满足这个条件,只用在哈密顿量里考虑是否加入$\operatorname{tr}_{R}\left(H_{S R} R_{0}\right)$即可。

虽然已经假设了初始时$S$和$R$无关,但由于相互作用的存在,之后的时刻里必然会出现新的关联,我们需要在这里引入Born近似。Born近似要求系统和环境间的相互作用是微弱的,这样一来系统就几乎不会对环境有反作用的影响。在这一近似下,环境的状态不随时间改变并且在相互作用中也一直保持不变。此外,Born近似也意味着在处理$\chi(t)$受到的环境影响时只包含$H_{SR}$的同阶量。这样一来,我们就可以把$\tilde{\chi}(t)$写作
\begin{equation}
\tilde{\chi}(t)=\tilde{\rho}(t) R_{0}+O\left(H_{S R}\right)
\end{equation}
这也可以等价的表述为:系统与环境的整体态$\tilde{\chi}(t)$的未来仅由系统态$\tilde{\rho}(t)$决定,而不依赖于环境态的历史。把Born近似应用于方程\eqref{MasterEq}并忽略掉$H_{S R}$的高阶项可以得到
\begin{equation}
\dot{\tilde{\rho}}=-\frac{1}{\hbar^{2}} \int_{0}^{t} d t^{\prime} \operatorname{tr}_{R}\left\{\left[\tilde{H}_{S R}(t),\left[\tilde{H}_{S R}\left(t^{\prime}\right), \tilde{\rho}\left(t^{\prime}\right) R_{0}\right]\right]\right\}
\label{MasterEqBorn}
\end{equation}

仔细观察上面的方程,我们会发现$\tilde{\rho}(t)$依赖于它自身的历史$\tilde{\rho}\left(t^{\prime}\right)$。换句话说,方程\eqref{MasterEqBorn}并不是Markov型的。Markov型的系统要求它的未来只取决于它现在的状态。所以我们另一个要做的主要近似就是把$\tilde{\rho}\left(t^{\prime}\right)$替换为$\tilde{\rho}(t)$,这样我们就得到了Born--Markov近似下的主方程:
\begin{equation}
\dot{\tilde{\rho}}=-\frac{1}{\hbar^{2}} \int_{0}^{t} d t^{\prime} \operatorname{tr}_{R}\left\{\left[\tilde{H}_{S R}(t),\left[\tilde{H}_{S R}\left(t^{\prime}\right), \tilde{\rho}(t) R_{0}\right]\right]\right\}
\end{equation}
Markov近似在物理上是合理的。具体来讲,系统$S$之所以依赖于它的历史是因为通过相互作用$H_{SR}$系统的早期态会映照在改变后的环境态上,而随后的早期态又会被改变后的环境态经过相互作用映照下来。但如果环境是一个一直处于热平衡的庞大系统,也就是热库的话,它就不会保持住来自$S$的改变太长时间。这样一来关键就在于比较热库的关联时间与系统$S$能发生明显变化的时间相比是否远远地小。下面来验证一下上述说法。

我们给定相互作用$H_{SR}$一个具体点的形式
\begin{equation}
H_{S R}=\hbar \sum_{i} s_{i} \Gamma_{i}
\label{HsrSpec}
\end{equation}
这里的$s_{i}$是系统$S$的Hilbert空间里的算符,$\Gamma_{i}$是热库$R$的Hilbert空间里的算符。然后依照上一节所述在相互作用表象下有
\begin{equation}
\begin{aligned}
\tilde{H}_{S R}(t) &=\hbar \sum_{i} e^{(i / \hbar)\left(H_{S}+H_{R}\right) t} s_{i} \Gamma_{i} e^{-(i / \hbar)\left(H_{S}+H_{R}\right) t} \\
&=\hbar \sum_{i}\left(e^{(i / \hbar) H_{S} t} s_{i} e^{-(i / \hbar) H_{S} t}\right)\left(e^{(i / \hbar) H_{R} t} \Gamma_{i} e^{-(i / \hbar) H_{R} t}\right) \\
&=\hbar \sum_{i} \tilde{s}_{i}(t) \tilde{\Gamma}_{i}(t)
\label{HsrSpec2}
\end{aligned}
\end{equation}
此时Born近似的主方程\eqref{MasterEqBorn}就变为了
\begin{align}
\dot{\tilde{\rho}}=&-\sum_{i, j} \int_{0}^{t} d t^{\prime} \operatorname{tr}_{R}\left\{\left[\tilde{s}_{i}(t) \tilde{\Gamma}_{i}(t),\left[\tilde{s}_{j}\left(t^{\prime}\right) \tilde{\Gamma}_{j}\left(t^{\prime}\right), \tilde{\rho}\left(t^{\prime}\right) R_{0}\right]\right]\right\} \notag \\
=&-\sum_{i, j} \int_{0}^{t} d t^{\prime}\left\{\tilde{s}_{i}(t) \tilde{s}_{j}\left(t^{\prime}\right) \tilde{\rho}\left(t^{\prime}\right) \operatorname{tr}_{R}\left[\tilde{\Gamma}_{i}(t) \tilde{\Gamma}_{j}\left(t^{\prime}\right) R_{0}\right]\right. \notag \\
&-\tilde{s}_{i}(t) \tilde{\rho}\left(t^{\prime}\right) \tilde{s}_{j}\left(t^{\prime}\right) \operatorname{tr}_{R}\left[\tilde{\Gamma}_{i}(t) R_{0} \tilde{\Gamma}_{j}\left(t^{\prime}\right)\right]-\tilde{s}_{j}\left(t^{\prime}\right) \tilde{\rho}\left(t^{\prime}\right) \tilde{s}_{i}(t) \notag \\
&\left.\times \operatorname{tr}_{R}\left[\tilde{\Gamma}_{j}\left(t^{\prime}\right) R_{0} \tilde{\Gamma}_{i}(t)\right]+\tilde{\rho}\left(t^{\prime}\right) \tilde{s}_{j}\left(t^{\prime}\right) \tilde{s}_{i}(t) \operatorname{tr}_{R}\left[R_{0} \tilde{\Gamma}_{j}\left(t^{\prime}\right) \tilde{\Gamma}_{i}(t)\right]\right\} \notag \\
=&-\sum_{i, j} \int_{0}^{t} d t^{\prime}\left\{\left[\tilde{s}_{i}(t) \tilde{s}_{j}\left(t^{\prime}\right) \tilde{\rho}\left(t^{\prime}\right)-\tilde{s}_{j}\left(t^{\prime}\right) \tilde{\rho}\left(t^{\prime}\right) \tilde{s}_{i}(t)\right]\left\langle\tilde{\Gamma}_{i}(t) \tilde{\Gamma}_{j}\left(t^{\prime}\right)\right\rangle_{R}\right. \label{MasterEqBorn2} \\
&\left.+\left[\tilde{\rho}\left(t^{\prime}\right) \tilde{s}_{j}\left(t^{\prime}\right) \tilde{s}_{i}(t)-\tilde{s}_{i}(t) \tilde{\rho}\left(t^{\prime}\right) \tilde{s}_{j}\left(t^{\prime}\right)\right]\left\langle\tilde{\Gamma}_{j}\left(t^{\prime}\right) \tilde{\Gamma}_{i}(t)\right\rangle_{R}\right\} \notag 
\end{align}
在上式的推导中我们用到了求迹运算的循环特性---$\operatorname{tr}\left( \hat{A} \hat{B} \hat{C} \right)=\operatorname{tr}\left(\hat{C} \hat{A} \hat{B} \right)= \operatorname{tr} \left( \hat{B} \hat{C} \hat{A} \right)$以及简化符号
\begin{equation}
\begin{aligned}
\left\langle\tilde{\Gamma}_{i}(t) \tilde{\Gamma}_{j}\left(t^{\prime}\right)\right\rangle_{R} &=\operatorname{tr}_{R}\left[R_{0} \tilde{\Gamma}_{i}(t) \tilde{\Gamma}_{j}\left(t^{\prime}\right)\right] \\
\left\langle\tilde{\Gamma}_{j}\left(t^{\prime}\right) \tilde{\Gamma}_{i}(t)\right\rangle_{R} &=\operatorname{tr}_{R}\left[R_{0} \tilde{\Gamma}_{j}\left(t^{\prime}\right) \tilde{\Gamma}_{i}(t)\right]
\end{aligned}
\end{equation}
热库的影响在方程\eqref{MasterEqBorn2}中体现为上式的两个关联函数。我们认为关联函数的变化在时间尺度上远快于$\tilde{\rho}(t)$的变化,理想情况下可以取
\begin{equation}
\left\langle\tilde{\Gamma}_{i}(t) \tilde{\Gamma}_{j}\left(t^{\prime}\right)\right\rangle_{R} \propto \delta\left(t-t^{\prime}\right)
\end{equation}
容易看出这样做的效果就等同于我们把$\tilde{\rho}(t^{\prime})$替换为$\tilde{\rho}(t)$。

\subsection{腔磁振子系统的主方程}
我们回到腔磁振子系统的哈密顿量,把热环境看作包含各种频率的谐振子的集合,并且认为系统内的光场和磁振子各自耦合于独立的热库
\begin{equation}
\begin{aligned}
H_S\equiv{}&\hbar\omega_{c}c^{\dag}c+\hbar\omega_{m}m^{\dag}m+\hbar gc^{\dag}m+\hbar gm^{\dag}c \\
&+i\hbar\Omega(c^{\dag}e^{-i\omega_{0}t}-ce^{i\omega_{0}t}) \\
H_{R}\equiv&\sum_{i}\hbar\omega_{i}a_{i}^{\dag}a_{i}+\sum_{j}\hbar\omega_{j}b_{j}^{\dag}b_{j} \\
H_{SR}\equiv&\sum_{i}g_{c,i}(c^{\dag}a_{i}+a_{i}^{\dag}c)+\sum_{j}g_{m,j}(m^{\dag}b_{j}+b_{j}^{\dag}m)
\label{HamCMRes}
\end{aligned}
\end{equation}
其中$a(a^{\dag})$,$b(b^{\dag})$分别为光子和谐振子热库的湮灭(产生)算符,$\omega_{i}$,$\omega_{j}$表示它们的频率,光子和谐振子与热库的耦合率分别为$g_{c,i}$,$g_{m,j}$。假设热库处于热平衡的温度为$T$,则其密度算符表示为
\begin{equation}
R_{0}=\prod_{i} e^{-\hbar \omega_{i} a_{i}^{\dag}a_{i} / k_{B} T}\left(1-e^{-\hbar \omega_{i} / k_{B} T}\right) \otimes \prod_{j} e^{-\hbar \omega_{j} b_{j}^{\dag}b_{j} / k_{B} T}\left(1-e^{-\hbar \omega_{j} / k_{B} T}\right)
\label{ThermalState}
\end{equation}
其中$k_{B}$表示玻尔兹曼常数,$R_{0}$即为满足玻色--爱因斯坦分布的密度算符。值得注意的是,关于\eqref{HamCMRes}中的$H_{SR}$,这里已经通过旋转波近似略去了粒子数不守衡的项,但是对于有着超强耦合的系统这一近似就不再成立,具体的讨论见相关研究\cite{}。

把哈密顿量\eqref{HamCMRes}对照上一小节\eqref{HsrSpec}的形式表示为
\begin{equation}
\begin{gathered}
s_{1}=c, \quad s_{2}=c^{\dagger}, \quad s_{3}=m, \quad s_{4}=m^{\dagger} \\
\Gamma_{1}= \sum_{i} g_{c,i} a_{i}^{\dag}, \quad \Gamma_{2}= \sum_{i} g_{c,i} a_{i}, \quad \Gamma_{3}= \sum_{j} g_{m,j} b_{j}^{\dag}, \quad \Gamma_{4}= \sum_{j} g_{m,j} b_{j}
\end{gathered}
\end{equation}
然后\eqref{HsrSpec2}中的算符就变为了
\begin{equation}
\begin{aligned}
&\tilde{s}_{1}(t)=e^{(i / \hbar) H_{S} t} c e^{-(i / \hbar) H_{S} t} \approx e^{i \omega_{c}c^{\dag}c t} e^{i \omega_{m}m^{\dag}m t} c e^{-i \omega_{c}c^{\dag}c t} e^{-i \omega_{m}m^{\dag}m t} = c e^{-i \omega_{0} t} \\
&\tilde{s}_{2}(t)=e^{(i / \hbar) H_{S} t} c^{\dagger} e^{-(i / \hbar) H_{S} t} \approx e^{i \omega_{c}c^{\dag}c t} e^{i \omega_{m}m^{\dag}m t} c^{\dagger} e^{-i \omega_{c}c^{\dag}c t} e^{-i \omega_{m}m^{\dag}m t} = c^{\dagger} e^{i \omega_{0} t} \\
&\tilde{s}_{3}(t)=e^{(i / \hbar) H_{S} t} m e^{-(i / \hbar) H_{S} t} \approx e^{i \omega_{c}c^{\dag}c t} e^{i \omega_{m}m^{\dag}m t} m e^{-i \omega_{c}c^{\dag}c t} e^{-i \omega_{m}m^{\dag}m t} = m e^{-i \omega_{0} t} \\
&\tilde{s}_{4}(t)=e^{(i / \hbar) H_{S} t} m^{\dagger} e^{-(i / \hbar) H_{S} t} \approx e^{i \omega_{c}c^{\dag}c t} e^{i \omega_{m}m^{\dag}m t} m^{\dagger} e^{-i \omega_{c}c^{\dag}c t} e^{-i \omega_{m}m^{\dag}m t} =m^{\dagger} e^{i \omega_{0} t}
\label{interactS}
\end{aligned}
\end{equation}
以及
\begin{equation}
\begin{aligned}
\tilde{\Gamma}_{1}(t) &=\exp \left(i \sum_{n}\omega_{n}a_{n}^{\dag}a_{n} t\right) \sum_{i} g_{c,i} a_{i}^{\dag} \exp \left(-i \sum_{m}\omega_{m}a_{m}^{\dag}a_{m} t\right) =\sum_{i} g_{c,i} a_{i}^{\dag} e^{i \omega_{i} t} \\
\tilde{\Gamma}_{2}(t) &=\exp \left(i \sum_{n}\omega_{n}a_{n}^{\dag}a_{n} t\right) \sum_{i} g_{c,i} a_{i} \exp \left(-i \sum_{m}\omega_{m}a_{m}^{\dag}a_{m} t\right) =\sum_{i} g_{c,i} a_{i} e^{-i \omega_{i} t} \\
\tilde{\Gamma}_{3}(t) &=\exp \left(i \sum_{n}\omega_{n}b_{n}^{\dag}b_{n} t\right) \sum_{j} g_{m,j} b_{j}^{\dag} \exp \left(-i \sum_{m}\omega_{m}b_{m}^{\dag}b_{m} t\right) =\sum_{j} g_{m,j} b_{j}^{\dag} e^{i \omega_{j} t} \\
\tilde{\Gamma}_{4}(t) &=\exp \left(i \sum_{n}\omega_{n}b_{n}^{\dag}b_{n} t\right) \sum_{j} g_{m,j} b_{j} \exp \left(-i \sum_{m}\omega_{m}b_{m}^{\dag}b_{m} t\right) =\sum_{j} g_{m,j} b_{j} e^{-i \omega_{j} t}
\label{interactGam}
\end{aligned}
\end{equation}
在\eqref{interactS}的推导中我们用到了$H_S$中耦合项与驱动项远远小于光子和谐振子谐振能量的假设,在下一章节的参数选取时我们会知道这在实验上是合理的,而在\eqref{interactGam}的推导中我们则用到了不同模式谐振子算符相互对易的特点。另外还需要注意的是,结合\eqref{ThermalState}与\eqref{interactGam}可以得到$\left\langle\tilde{\Gamma}_{1}(t)\right\rangle_{R_0}=\left\langle\tilde{\Gamma}_{2}(t)\right\rangle_{R_0}=\left\langle\tilde{\Gamma}_{3}(t)\right\rangle_{R_0}=\left\langle\tilde{\Gamma}_{4}(t)\right\rangle_{R_0}=0$,也就是说式\eqref{InitialAssumption}的假设是成立的。

\section{Fokker-Planck方程}


\section{随机微分方程}