% !TeX root = ../main.tex
% -*- coding: utf-8 -*-

\chapter{理论方法}
\label{ch3}

\section{主方程}
为了在模型中加入环境的影响,我们可以假设系统$S$和热库$R$的耦合具有如下的哈密顿量形式:
\begin{equation}
H=H_{S}+H_{R}+H_{S R}
\end{equation}
这里$H_{S}$和$H_{R}$分别是系统和热库的哈密顿量,$H_{SR}$为相互作用的哈密顿量。我们可以先不拘泥于热库哈密顿量的具体形式,只需要知道他可以表示为温度和能量的函数。接下来我们的目标是在无需关注热库具体状态的情形下研究混合体系$S \otimes R$中目标系统$S$的演化。为了做到这一点,需要用到密度矩的工具。我们用$\chi(t)$表示$S \otimes R$的密度算符,定义约化密度算符为
\begin{equation}
\rho(t) \equiv \operatorname{tr}_{R}[\chi(t)]
\end{equation}


\section{Fokker-Planck方程}


\section{随机微分方程}