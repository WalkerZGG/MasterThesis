% !TeX root = ../main.tex
% -*- coding: utf-8 -*-

\chapter{理论方法}
\label{ch3}

\section{主方程}

\subsection{Liouville--von Neumann方程的解}
为了在模型中加入环境的影响,我们可以假设系统$S$和外部环境$R$的耦合具有如下的哈密顿量形式:
\begin{equation}
\mathcal{H}=H_{S}+H_{R}+H_{S R}
\label{Hsr}
\end{equation}
这里$H_{S}$和$H_{R}$分别是系统和环境的哈密顿量,$H_{SR}$为相互作用的哈密顿量。我们可以先不拘泥于环境哈密顿量的具体形式,只需要知道他可以表示为温度和能量的函数。我们的目标是在无需关注环境具体状态的情形下研究混合体系$S \otimes R$中目标系统$S$的演化。为了做到这一点,需要用到密度算符的工具。我们用$\chi(t)$表示$S \otimes R$的密度算符并对环境的部分求迹,定义约化密度算符为
\begin{equation}
\rho(t) \equiv \operatorname{tr}_{R}[\chi(t)]
\end{equation}
显然,如果我们有了约化密度算符$\rho(t)$,很容易就能算出系统$S$的Hilbert空间中任一算符$\hat{O}$的平均,而不需要知道$\chi(t)$的具体形式:
\begin{equation}
\langle\hat{O}\rangle=\operatorname{tr}_{S \otimes R}[\hat{O} \chi(t)]=\operatorname{tr}_{S}\left\{\hat{O} \operatorname{tr}_{R}[\chi(t)]\right\}=\operatorname{tr}_{S}[\hat{O} \rho(t)]
\end{equation}
接下来我们要做的是得到$\rho(t)$演化方程。

%Schr\"{o}dinger
已知$\chi$的Liouville--von Neumann方程为
\begin{equation}
\dot{\chi}=\frac{1}{i \hbar}[\mathcal{H}, \chi]
\label{LMEq1}
\end{equation}
这里可以把$\mathcal{H}$中的$H_{S}+H_{R}$看作主要部分,$H_{SR}$当成次要部分,转入相互作用绘景下来方便我们的处理:
\begin{equation}
\tilde{\chi}(t) \equiv e^{(i / \hbar)\left(H_{S}+H_{R}\right) t} \chi(t) e^{-(i / \hbar)\left(H_{S}+H_{R}\right) t}
\end{equation}
结合\eqref{Hsr}、\eqref{LMEq1}与上式后可以得到
\begin{align}
\dot{\tilde{\chi}} &=\frac{i}{\hbar}\left(H_{S}+H_{R}\right) \tilde{\chi}-\frac{i}{\hbar} \tilde{\chi}\left(H_{S}+H_{R}\right)+e^{(i / \hbar)\left(H_{S}+H_{R}\right) t} \dot{\chi} e^{-(i / \hbar)\left(H_{S}+H_{R}\right) t} \notag \\
&=\frac{1}{i \hbar}\left[\tilde{H}_{S R}(t), \tilde{\chi}\right]
\label{LMEq2}
\end{align}
% oOSRHH{\ttfamily oOSRHH}{\sffamily oOSRHH}
其中$\tilde{H}_{S R}(t)$是显含时间的量:
\begin{equation}
\tilde{H}_{S R}(t) \equiv e^{(i / \hbar)\left(H_{S}+H_{R}\right) t} H_{S R} e^{-(i / \hbar)\left(H_{S}+H_{R}\right) t}
\end{equation}
对方程\eqref{LMEq2}直接积分可以得到
\begin{equation}
\tilde{\chi}(t)=\chi(0)+\frac{1}{i \hbar} \int_{0}^{t} d t^{\prime}\left[\tilde{H}_{S R}\left(t^{\prime}\right), \tilde{\chi}\left(t^{\prime}\right)\right]
\end{equation}
把上式回代到方程\eqref{LMEq2}的等号右侧:
\begin{equation}
\dot{\tilde{\chi}}=\frac{1}{i \hbar}\left[\tilde{H}_{S R}(t), \chi(0)\right]-\frac{1}{\hbar^{2}} \int_{0}^{t} d t^{\prime}\left[\tilde{H}_{S R}(t),\left[\tilde{H}_{S R}\left(t^{\prime}\right), \tilde{\chi}\left(t^{\prime}\right)\right]\right]
\label{LMExpd2}
\end{equation}
以上的推导都是严格的,我们甚至可以重复\eqref{LMEq2}到\eqref{LMExpd2}的过程来得到无限多的积分项,但是这里做到二阶就足够引入我们下面的近似了。

\subsection{Born--Markov近似}
假设$t=0$时的系统$S$和环境$R$之间没有关联但是相互作用已然存在,我们可以把$\chi(0)=\tilde{\chi}(0)$表示为
\begin{equation}
\chi(0)=\rho(0) R_{0}
\end{equation}
其中$R_{0}$是初始时刻环境的密度矩阵。记$\tilde{\chi}(t)$求迹后的密度矩阵为$\tilde{\rho}(t)$:
\begin{equation}
\operatorname{tr}_{R}[\tilde{\chi}(t)]=e^{(i / \hbar) H_{S} t} \rho(t) e^{-(i / \hbar) H_{S} t} \equiv \tilde{\rho}(t)
\end{equation}
对方程\eqref{LMExpd2}进行求迹运算后我们就可以得到如下的主方程:
\begin{equation}
\dot{\tilde{\rho}}=-\frac{1}{\hbar^{2}} \int_{0}^{t} d t^{\prime} \operatorname{tr}_{R}\left\{\left[\tilde{H}_{S R}(t),\left[\tilde{H}_{S R}\left(t^{\prime}\right), \tilde{\chi}\left(t^{\prime}\right)\right]\right]\right\}
\label{MasterEq}
\end{equation}
这里我们已经通过假设$\operatorname{tr}_{R}\left[\tilde{H}_{S R}(t) R_{0}\right]=0$消去了\eqref{LMExpd2}中的第一项。这项假设需要相互作用$H_{S R}$在$R_{0}$态下的平均值为零,我们总是可以满足这个条件,只用在哈密顿量里考虑是否加入$\operatorname{tr}_{R}\left(H_{S R} R_{0}\right)$即可。

虽然已经假设了初始时$S$和$R$无关,但由于相互作用的存在,之后的时刻里必然会出现新的关联,我们需要在这里引入Born近似。Born近似要求系统和环境间的相互作用是微弱的,这样一来系统就几乎不会对环境有反作用的影响。在这一近似下,环境的状态不随时间改变并且在相互作用中也一直保持不变。此外,Born近似也意味着在处理$\chi(t)$受到的环境影响时只包含$H_{SR}$的同阶量。这样一来,我们就可以把$\tilde{\chi}(t)$写作
\begin{equation}
\tilde{\chi}(t)=\tilde{\rho}(t) R_{0}+O\left(H_{S R}\right)
\end{equation}
这也可以等价的表述为:系统与环境的整体态$\tilde{\chi}(t)$的未来仅由系统态$\tilde{\rho}(t)$决定,而不依赖于环境态的历史。把Born近似应用于方程\eqref{MasterEq}并忽略掉$H_{S R}$的高阶项可以得到
\begin{equation}
\dot{\tilde{\rho}}=-\frac{1}{\hbar^{2}} \int_{0}^{t} d t^{\prime} \operatorname{tr}_{R}\left\{\left[\tilde{H}_{S R}(t),\left[\tilde{H}_{S R}\left(t^{\prime}\right), \tilde{\rho}\left(t^{\prime}\right) R_{0}\right]\right]\right\}
\end{equation}


\section{Fokker-Planck方程}


\section{随机微分方程}