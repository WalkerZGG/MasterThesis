% !TeX root = ../main.tex
% -*- coding: utf-8 -*-


\begin{zhaiyao}
以量子计算为代表的量子信息处理领域是目前物理学研究中最为火热的方向之一。由于量子计算机需要对信息进行频繁的读写操作,能够实现量子计算的物理系统无一不是有着强耦合的开放量子系统。而近些年来微波腔与YIG球之间强耦合的实现使得腔磁振子系统越发受到人们的青睐,在这个系统中磁性材料里大量自旋粒子的集体运动所产生的磁振子模式可以和微腔中的束缚电磁场相互作用,许多研究人员在实验和理论上对它进行了研究并取得了一系列吸引人的成果。毫无疑问腔磁振子系统为强耦合量子系统的研究开辟了新的道路,并有望在量子混合系统、量子精密测量等领域实现新的应用。

以往的研究工作中对腔磁振子系统的理解多是建立在半经典理论或是基于量子朗之万方程的输入--输出理论上,以方便与实验结果匹配。但是使用这些方法可能会遗漏掉腔磁振子系统中的某些量子效应,而且目前的研究中仍旧缺乏对此系统里量子态以及关联性质的详细讨论。本文的研究工作从开放量子系统的角度出发,在腔中光子与磁振子耦合的哈密顿量里加入系统与环境的耦合,并进一步将得到的主方程转换为等价的Fokker-Planck 方程来讨论腔磁振子系统里的稳态以及动力学。在方程的求解中,我们导出了一组级联方程来得到腔磁振子系统的稳态谱。针对基于实验的强耦合、MIT、Purcell这三种耦合参数,我们计算了符合实验观测的平均粒子谱以及揭示出系统中相干竞争机制的二阶关联函数谱。我们还使用了随机微分方程的方法来模拟系统中量子态的演化,在有限次抽样的平均中得到了腔磁振子系统中关联函数的动力学行为,并展示了强耦合下的Rabi振荡以及MIT、Purcell参数下振荡的衰退。对于目前实验中能够测量的量,我们的计算结果可以很好地描述这些测量值的特征。而对于当前实验中缺乏的二阶关联函数的观测,我们给出了可能的实验设置,并把理论参数转换为实验值来预测相应条件下的结果。此外,我们的方法还可以进一步拓展到非线性哈密顿量中对系统高阶量子关联的求解,用来预测以往研究中不曾涉及过的新的量子特性。
\end{zhaiyao}




\begin{guanjianci}
腔磁振子系统;强耦合;开放量子系统;量子关联;二阶关联函数
\end{guanjianci}



\begin{abstract}

Quantum information processing represented by quantum computing is one of the hottest fields in physics research nowadays. Since quantum computers need to perform frequent read and write operations on information, all physical systems that can realize quantum computing are open quantum systems with strong coupling. In recent years, the realization of strong coupling between microwave cavity and YIG sphere makes the cavity magnonics system more and more popular. In this system, the elementary excitations in magnetic materials named magnons strongly interact with the confined electromagnetic fields in microcavity. Many researchers have studied it experimentally and theoretically and obtained a series of attractive results. There is no doubt that the cavity magnonics system opens up a new path for the research of strong coupling quantum systems, and is expected to achieve new applications in hybrid quantum systems, quantum precision measurement and other fields.

Previous research work in the understanding of the cavity magnonics system is founded on the semi-classical theory or quantum Langevin equation based input--output theorem, for matching with the experimental results. However, using these methods may miss some quantum effects in the cavity magnonics system, and there is still a lack of detailed discussion on the quantum states and correlation properties of the system. In this paper, from the perspective of open quantum system, the coupling between system and environment is added into the Hamiltonian of cavity magnonics, and the obtained master equation is further converted into the equivalent Fokker-Planck equation to discuss the steady state and dynamics of the system. In solving this equation, we have derived a set of hierarchical equations to obtain the steady state spectra of the cavity magnonics system. Under the three coupling parameter regimes of strong coupling, MIT and Purcell, we calculate the average particle number spectra consistent with the experimental observations and the the second-order correlation function spectra that reveal the mechanism of coherence competition in the system. We have also used the stochastic differential equation method to simulate the evolution of the quantum state in the system, obtain the dynamic behavior of the correlation functions with the average of trajectory samples, and show the Rabi oscillation under strong coupling and the decay of oscillation under MIT and Purcell parameters. For the quantities that can be measured in the present experiments, our calculation results can well describe the characteristics of these measured values. For the observation of the second-order correlation function lacking in the current experiment, possible experimental settings are proposed, and the parameters used in theory are converted to experimental values to predict the results under the corresponding conditions. In addition, our method can be further extended to solve higher-order quantum correlations of systems with nonlinear Hamiltonian, which can be used to predict new quantum properties which have not been involved in previous studies.

\end{abstract}



\begin{keywords}
cavity magnonics; strong coupling; open quantum system; quantum correlation; second-order correlation function
\end{keywords} 