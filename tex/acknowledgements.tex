% !TeX root = ../main.tex
% -*- coding: utf-8 -*-

%\makeschapterhead{致谢}
\chapter*{致谢}
{
	\ttfamily \fontsize{12bp}{16bp}\selectfont
	我依旧清楚地记得三年以前的晚上,刚完成本科毕业论文的时候,心头仍有千言万语萦绕其上。然而两个月前,面对提交完的本论文初稿,我心中却没有释怀,更没有喜悦,只是空荡荡的,一句话也说不出来。现在想起来,或许是因为疫情被封在学校里太久,或许是因为渐渐中年后身体状态的变差,又或许是因为生活习惯的不注意,总之,我当时似乎陷入了一种无法对任何事情打起精神的状态。如今腹中依然提不起太多笔墨,但还是可以稍微回顾一下硕士这三年的。
	
	记忆中第一年刚步入校园的时候,正值南开百年,我带着尚未褪去的青春烂漫,怀揣着热情和希望开始了人生中新一阶段的学习和成长。但是疫情的突然到来,裹挟了后面两年多的读书生活,却也让我能把更多的时间和精力放在科研工作之上。在这三年间,我经历过冥思苦想的顿悟,也遇到过思而不得的挫败,有过喜悦,也有过低落。在一次又一次的求索中,我自知没有泡利、朗道那样的天赋神通,只能在被敲打的过程中慢慢摸上门道,不知不觉间,先贤的影子渐行渐远,转眼间竟成了天上的启明星。
	
	我没办法触及头顶的星星。但却幸运的有着眼前的引路灯。感谢我的学术导师王永老师在这三年里的陪伴,是您严谨的治学态度鞭策着我在学术道路上砥砺前行,使这样的我也能够收获初步的成果,顺利入门科研。感谢共同合作导师钱晓锋老师,您给了我学习交流的机会,期待未来五年与您的相处。感谢课题组内已经离去的郑亚卿师姐、周丽萍师姐、孙华筝师姐、王丹师姐、张萌师姐、李雪梅师姐、秦任师兄、刘灿师姐,你们对刚进组时一无所知的我的关照让我受益良多。虽然组里没有和我同级的同学,但是师兄弟、师姐妹间的相处却并没有让我感到隔阂。感谢仍在组内的王安州师弟、许白师弟、金茹芳师妹、贾宇桢师弟、郭云清师妹,和你们一起相处的时间总是让我感觉到欢乐,希望以后再相遇时还能有机会把酒言欢、共叙桑麻。
	
	虽然不是一个课题组,但在同一间办公室里朝夕相处的胡振芃老师课题组的同学们,也要多多感谢你们的照顾。哪怕是许多年以后,我想我也会记得在我入学前和毕业后仿佛一直都在的郑彩艳师姐,会记得嘴里家常经常停不下来的庄新莹师姐,会记得研一夏季总是一起最晚离开办公室的刘佳琪师妹。感谢高乾师兄、孔龙娟师姐、张丽芙师姐、褚红琴师姐、宋文丽师姐、王涵师弟、李琢瑶师妹,和你们一起共事的点点滴滴就像沙滩上的珍珠一样点缀了我的整个研究生生活。
	
	当然最需要感谢的还是我的父母,每当我心有畏缩的时候,家人永远是那个最温暖的港湾。因为有您二老的支持和关爱,我才能顺利完成学业,祝愿父母亲身体健康、长命百岁。
	
	最后要感谢我的祖国,感谢国家对研究生的补助帮我减轻了家里的负担,感谢国家的安定和建设规划让无力赚钱的学生也能安稳读书,祝愿国家顺利完成产业升级,早日实现中华民族的伟大复兴。
	
	\vspace{3em}
	\begin{flushright}
	2022年6月

	于八里台\hspace*{1em}
	\end{flushright}
}