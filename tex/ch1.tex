% !TeX root = ../main.tex
% -*- coding: utf-8 -*-

\chapter{绪论}
\label{ch1}

\section{课题背景}
1946年,世界上第一台电子计算机ENIAC在美国诞生,随后的短短30年间,电子计算机的发展便经历了“电子管”、“晶体管”、“集成电路”、“大规模集成电路”四代变革,计算机体积在不断缩小的同时计算能力却提高了6个数量级。尤其是集成技术的发展,使得一小块半导体芯片上可以容纳上百万个晶体管,并且这个数量随着产品的更新迭代还在逐年增长。当时,著名的摩尔定律指出:集成电路上可以容纳的晶体管数目在大约每经过24个月便会增加一倍。摩尔定律相当精准地预测了之后40年计算机性能的发展速度,但是随着晶体管密度的不断提高,元器件尺寸也在不断缩小,当晶体管尺寸小到接近原子尺寸时芯片性能将难以提升。2020年,台积电宣布其5纳米工艺制程芯片已进入批量生产,并且3纳米工艺也将在2021年面世。可以预见,在不久的未来摩尔定律将完全失效,经典计算机的性能也将遇到瓶颈。

为了延续甚至超越摩尔定律的辉煌,人们需要一种在底层原理上不同于经典计算机的设计,近些年来经过科研人员的不断探索,量子计算机被认为是一种能达成这一目标的途径。早在1982年,诺贝尔物理学奖得主 Richard Feynman 就提出了量子计算机的构想:既然物理世界是用量子力学的语言描述的,那么就应该用遵从量子力学原理的计算机来模拟真实世界。1985年,牛津大学的 David Deutsch 指出可以利用相干叠加原理实现通用量子计算。1994年, Peter Shor 证明了运行于量子计算机中的算法可以实现对大数质因子的快速分解,这一算法被认为可以轻易摧毁现有的公钥加密系统并且是经典计算机所无法企及的。1998年,美国洛斯阿拉莫斯国家实验室与麻省理工、加州伯克利大学合作造出了第一台基于有机分子的2比特量子计算机并验证了量子计算的一些基本原理。此后的十几年间,多种不同的量子计算机制造方案被提出并得到了实现,包括离子阱、量子点、超导量子比特等。但这些技术都面临着有效量子比特无法轻易做大的问题,其中主要的困难在于量子叠加态是十分脆弱的,极易因环境而发生退相干,因此必须在极短的相干时间内完成量子信息的存储、传输和处理,这就要求量子比特与操控它的光场间的耦合远大于各自的耗散,即实现所谓的强耦合。2001年,段路明等人提出使用原子群可以实现正比于粒子数平方根的强耦合,并设计了用于长程量子中继器和寄存器的方案。随后,超冷原子群与光学腔间的强耦合被实验所证实。2009年,Imamo\u{g}lu 指出此前的耦合系统都是光与物质间通过电偶极相互作用的耦合,即腔量子电动力学(Cavity QED)的框架下,而光与自旋粒子群间的磁偶极相互作用能够更轻易的提高耦合率。2014年, M. E. Tobar 与 Y. Nakamura 两个组各自独立的实现了微波腔与钇铁石榴石(YIG)球间的强耦合,同年, H. Tang 课题组将强耦合做到了室温。2015年, C.-M. Hu 等人对YIG球中自旋的集体运动,即磁振子模式,进行了间接的电学测量。2016年, J. Q. You 等人实验上做出了磁振子的Kerr非线性效应。此后, H. Tang 和 Y. Nakamura 小组又各自将腔磁振子系统与超导量子比特和声子模进行耦合,朝混合量子系统的方向进行了探究。另一方面, M. E. Tobar 小组进一步的将腔与YIG间的耦合率提升到GHz,实现了所谓的超强耦合。2020年, G. Ruoso 小组利用腔磁振子系统实现了磁场的超高精度测量。

可以看到,近几年腔磁振子系统在实验上的研究进展非常迅猛,与此同时,对这个系统在理论上的探讨也有很多吸引人的发现,比如量子纠缠,非厄米物理,拓扑性质,非互易性等。但是以往对此系统的研究用的多是半经典理论或者是结合实验的输入-输出理论,无法直接了解到系统中量子态的性质。因此我们接下来的工作将使用传统量子光学中的方法来探究腔磁振子系统中包括量子关联在内的一些量子性质。

\section{常用内容}
